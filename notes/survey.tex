\chapter{Survey}


\section{T-systems, networks and dimers \cite{di2014tystemsnetworksdimers}	}
\subsection{Excerpts}
\begin{itemize}
	\item  In the case of type A, the T-system equation is also known as the octahedron recurrence, and appears to be central in a number of combinatorial objects, such as the lambda-determinant and the Alternating Sign Matrices [24][8], the puzzles for computing Littlewood-Richardson coefficients [20], generalizations of Coxeter-Conway frieze patterns [5][1][3], and the domino tilings of the Aztec diamond [12][25]
\end{itemize}

\section{Q-systems as Cluster Algebras {II}: Cartan Matrix of Finite Type and the Polynomial Property \cite{Di_Francesco_2009}}
A new interpretation for the T-system arose from realizing that the corresponding discrete evolution could be viewed as a particular mutation in a suitably defined cluster algebra

\section{Arctic curves of the octahedron equation \cite{di2014arctic}}


\section{Double-dimers, the Ising model and the hexahedron recurrence \cite{kenyon2013double}}

\section{Perfect Matchings and the Octahedron Recurrence \cite{speyer2004perfect}}