\chapter{Survey}

\section{Local dimer dynamics in higher dimensions \cite{hartarsky2023local}}

\subsection{Conjectures}

\begin{conjecture}\label{conj-dynamics-1}
	For all $3$-dimensional shapes $\mathbf{n}$, the graphs $\cD(\bQ_\mathbf{n}^3)$ is connected.
\end{conjecture}

\subsection{Theorems}
\begin{theorem}[Extraction of a dense \bQ]\label{thm-dynamics-3}
	Let $d \geq 2$ and $\bfn$ be a $d$-dimensional shape. Then, for any dimer configuration $D$ on $\bQ_\bfn^d$, there exists $\bfx \in \bZ^d$ such that th eunit cube $\bfx + \bQ^d \subseteq \bQ_\bfn^d$ contains at least $2^{d - 2} + 1$ dimers in $D$.
\end{theorem}

\begin{theorem}[Degree and component size]\label{thm-dynamics-4}
	Fix $d \geq 3$ and an even positive integer $n$. The minimum degree of $\cD_{2d - 1}(\bQ^d_n)$ is at least $n^{d - 2}/(320d^6)$ and each connected component contains at least $2^{n^{d - 2}/(320d^6)}$ dimer configurations.
\end{theorem}

\begin{theorem}[Degree and component size in $\bQ^d$]\label{thm-dynamics-5}
	Fix $d \geq 3$. The minimum degree of $\cD_{2d - 1}(\bQ^d)$ is at least $2^d/(d^4)$ and each connected component contains at least $2^{2^d/(d^4)}$ dimer configurations.
\end{theorem}

\begin{theorem}[Ergodicity on $\bQ^d$]\label{thm-dynamics-6}
	For every $d \geq 2$, the graph $\cD_{2d - 2}(\bQ^d)$ is connected and has diameter at most $(d - 1) 2^{d - 1}$.
\end{theorem}

\begin{theorem}[Diameter lower bound]\label{thm-dynamics-7}
	For all $d, l, n \geq 2$ with $n$ even, the graph $\cD_{l}(\bQ^d_n)$ has diameter at least $$\frac{n^{d - 1} (n^2 - 1)}{6l^2}.$$
\end{theorem}

\begin{theorem}[Ergodicity on $\bT_{m,n}$]\label{thm-dynamics-8}
	For all positive integers $m, n$ with $mn$ even, the graph $\cD_{3}(\bT_{m,n})$ is connected and has diameter at most $2mn$.
\end{theorem}

\begin{lemma}\label{lemma-dynamics-10}
	Fix $d \geq 3$, $n \geq 4$. Then, there are at least $n^{d-2}/(20d^2)$ authorised vertices in any dimer configuration on $\bQ_n^d$. 
\end{lemma}

\section{Uniformly positive correlations in the dimer model and
	phase transition in lattice permutations in $\mathbb{Z}^d$, $d > 2$, via reflection positivity \cite{taggi2019uniformly}}

\section{T-systems, networks and dimers \cite{di2014tystemsnetworksdimers}	}
\subsection{Excerpts}
\begin{itemize}
	\item  In the case of type A, the T-system equation is also known as the octahedron recurrence, and appears to be central in a number of combinatorial objects, such as the lambda-determinant and the Alternating Sign Matrices [24][8], the puzzles for computing Littlewood-Richardson coefficients [20], generalizations of Coxeter-Conway frieze patterns [5][1][3], and the domino tilings of the Aztec diamond [12][25]
\end{itemize}

\section{Q-systems as Cluster Algebras {II}: Cartan Matrix of Finite Type and the Polynomial Property \cite{Di_Francesco_2009}}
A new interpretation for the T-system arose from realizing that the corresponding discrete evolution could be viewed as a particular mutation in a suitably defined cluster algebra

\section{Arctic curves of the octahedron equation \cite{di2014arctic}}


\section{Double-dimers, the Ising model and the hexahedron recurrence \cite{kenyon2013double}}

\section{Perfect Matchings and the Octahedron Recurrence \cite{speyer2004perfect}}

